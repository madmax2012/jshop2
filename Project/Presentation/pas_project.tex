\documentclass{beamer}

\usepackage{graphics}
\usetheme{metropolis}           % Use metropolis theme
\usepackage{color}
\usepackage{amssymb}
\usepackage{amsmath}
\usepackage{mathtools}
\usepackage{amstext}
\usepackage{blindtext}
\usepackage{float}
\usepackage{csquotes}
\usepackage{todonotes}
\usepackage{afterpage}
\usepackage{caption}
\usepackage{framed}

%\usepackage[
%backend=bibtex,
%style=nature 	,
%]{biblatex}	

%\addbibresource{library.bib}


\title{Planning and Scheduling at home project}
\date{\today}
\author{\'{A}ngela Patricia Enr\'{i}quez G\'{o}mez, Ethan Oswald Massey, Maximilian Mensing}
\institute{BRSU}
\begin{document}
\maketitle


\section{Introduction}

\begin{frame}{\textsl{ Task }}

\begin{itemize}
	\item 
\end{itemize}

\end{frame}

\begin{frame}{\textsl{M\"ogliche Abwandlungen von Pair Programming}}
	  
	\begin{itemize}
		\item 
	\end{itemize}


\begin{alertblock}{Conclusion}
	Abwandlung bei uns: Pairing
\end{alertblock}
\end{frame}


\section{Implementation}

\begin{frame}{\textsl{ Was m\"ochten wir mit "Pairing" erreichen}}
	\begin{itemize}
		\item 
	\end{itemize}

\end{frame}




\begin{frame}{Guidelines}
\begin{itemize}
	\item
\end{itemize}
\end{frame}


\section{Diskussion}

\begin{frame}{Offene Punkte}
\begin{itemize}

	\item 
\end{itemize}
\end{frame}




\begin{frame}{}

\end{frame}


%%%%%%%%%%%%%%%%%%%  torcs -> parameter -> genom -> optimisation problem 
%\printbibliography
%
\end{document}