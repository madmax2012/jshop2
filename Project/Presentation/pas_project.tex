\documentclass{beamer}

\usepackage{graphics}
\usetheme{metropolis}           % Use metropolis theme
\usepackage{color}
\usepackage{amssymb}
\usepackage{amsmath}
\usepackage{mathtools}
\usepackage{amstext}
\usepackage{blindtext}
\usepackage{float}
\usepackage{csquotes}
\usepackage{todonotes}
\usepackage{afterpage}
\usepackage{caption}
\usepackage{framed}

%\usepackage[
%backend=bibtex,
%style=nature 	,
%]{biblatex}	

%\addbibresource{library.bib}


\title{Planning and Scheduling at home project}
\date{\today}
\author{\'{A}ngela Patricia Enr\'{i}quez G\'{o}mez, Ethan Oswald Massey, Maximilian Mensing}
\institute{BRSU}
\begin{document}
\maketitle


\section{Introduction}

\begin{frame}{\textsl{ Task }}

Storing Groceries

\begin{itemize}
	\item \textbf{Case 1}: Everything is known. One object is on the table and has to be placed at any shelf after the cupboard has been opened.
	\item \textbf{Case 2}: The amount of Objects, the table and the cupboard have to be located. The objects have to be placed on the shelves.
	\item \textbf{Case 3}: As case 2 - In addition to not knowing the number of objects, the objects themself are also unknown.
	\item \textbf{Case 4}: In addition to case 3 the cupboard is unknown and has to be explored. The items have to be put in different categories and sorted by category on the shelf.
\end{itemize}

\end{frame}

\begin{frame}{\textsl{Selection of the Planner}}
	  
\begin{figure}
	\centering
	\includegraphics[width=0.99\linewidth]{jshop_comparision}
	\label{fig:kim}
\end{figure}

\end{frame}


\begin{frame}{\textsl{Using JSHOP2}}
	\begin{itemize}
		\item Get jshop from \url{https://github.com/mas-group/jshop2}
		\item Set environment \textbf{{\tiny export CLASSPATH="`pwd`/bin.build/JSHOP2.jar:`pwd`/antlr.jar:."}}
		\item Compile using \textbf{{\tiny make c}}
		\item Run by calling \textbf{{\tiny make problem1/2/3/4}}
	\end{itemize}

\end{frame}




\section{Limitations and Issues}

\begin{frame}{Limitations}
\begin{itemize}
	\item Without sufficient prior information the planner is not able to classify objects
	\item In problem 4, if the shelves don't have example objects  the planner has problems putting categories for them.
	\item Executing with java -ra generates all possible plans 
	\begin{itemize}
		\item High space and time complexity
		\item Maximum of 4 objects to avoid outOfMemoryError
	\end{itemize} 
\end{itemize}
\end{frame}


\section{Diskussion}

\begin{frame}{Offene Punkte}
\begin{itemize}

	\item 
\end{itemize}
\end{frame}




\begin{frame}{}

\end{frame}


%%%%%%%%%%%%%%%%%%%  torcs -> parameter -> genom -> optimisation problem 
%\printbibliography
%
\end{document}